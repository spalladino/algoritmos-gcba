\documentclass[11pt,a4paper]{article}
\usepackage{a4wide}
\usepackage{verbatim}
\usepackage{amssymb}
\usepackage{amsthm}
\usepackage{amsmath}
\usepackage{enumerate}
\usepackage{latexsym}
\usepackage{algorithm}
\usepackage[noend]{algorithmic}
\usepackage[latin1]{inputenc}

                 
\theoremstyle{definition}\newtheorem{exercise}{Ejercicio}                 
                                                      
\title{Pr�ctica 2\\
Introducci�n a los Algoritmos}
\author{Algoritmos y Estructuras de Datos\\
Programaci�n Estructurada}
\date{}

\setlength{\topmargin}{-2cm}
\setlength{\oddsidemargin}{-0.5cm}
\setlength{\evensidemargin}{-0.5cm}


%%%%%%%%%%%%%%%%%%%%%%%%%%%%%%%%%%%%%%%%%%%%%%%%%%%%%%%%%%%%%%%%%%%%%%%%%%%%%%%%%%%%%%%%%%%%%%%%%%%%%%%%%%%%%%%%%%%%%%%%

\begin{document}

\maketitle

\begin{exercise}
Para cada uno de los siguientes problemas, dar una especificaci�n y dar un algoritmo en pseudoc�digo que lo resuelva.
\begin{enumerate}
\item Convertir un lapso de tiempo expresado en d�as, horas, minutos y segundos a uno expresado en segundos.
\item Dada una fecha (expresada como la cantidad de d�as que pasaron desde el primero de enero), calcular cu�ntos d�as \textit{faltan} para que empiece el verano de ese a\~no (es decir, para el 21 de diciembre). Imponga la precondici�n que considere necesaria.
\end{enumerate}
\end{exercise}


\end{document}

