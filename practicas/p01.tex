\documentclass[11pt,a4paper]{article}
\usepackage{a4wide}
\usepackage{verbatim}
\usepackage{amssymb}
\usepackage{amsthm}
\usepackage{amsmath}
\usepackage{enumerate}
\usepackage{latexsym}
\usepackage{algorithm}
\usepackage[noend]{algorithmic}
\usepackage[latin1]{inputenc}

                 
\theoremstyle{definition}\newtheorem{exercise}{Ejercicio}                 
                                                      
\title{Pr�ctica 1\\
Introducci�n a los Algoritmos}
\author{Algoritmos y Estructuras de Datos\\
Programaci�n Estructurada}
\date{}

\setlength{\topmargin}{-2cm}
\setlength{\oddsidemargin}{-0.5cm}
\setlength{\evensidemargin}{-0.5cm}


%%%%%%%%%%%%%%%%%%%%%%%%%%%%%%%%%%%%%%%%%%%%%%%%%%%%%%%%%%%%%%%%%%%%%%%%%%%%%%%%%%%%%%%%%%%%%%%%%%%%%%%%%%%%%%%%%%%%%%%%

\begin{document}

\maketitle

\begin{exercise}
Responda las siguientes preguntas:
\begin{enumerate}
\item �Qu� es una instrucci�n?
\item �Qu� es un programa?
\item �Qu� es un algoritmo?
\item �Qu� diferencia hay entre un programa y un algoritmo?
\end{enumerate}
\end{exercise}

\begin{exercise}
Indique cu�l de los siguientes programas puede considerarse un algoritmo y cu�l no, justificando:
\begin{itemize}
\item \begin{verbatim}
Contar colectivos I
-------------------
1. Ir a la parada m�s cercana de colectivos
2. Contar cu�ntos colectivos pasan por dicha parada
\end{verbatim}
\item \begin{verbatim}
Contar colectivos II
--------------------
1. Ir a la parada m�s cercana de colectivos
2. Contar cu�ntos colectivos pasan por dicha parada durante una hora
\end{verbatim}
\end{enumerate}
\end{exercise}

\begin{exercise}
Responda las siguientes preguntas:
\begin{enumerate}
\item �Qu� es la precondici�n de un programa?
\item �Qu� es la poscondici�n de un programa?
\item �Cu�ndo un programa es correcto?
\end{enumerate}
\end{exercise}

\begin{exercise}
Para cada una de las siguientes tareas, d\'e una especificaci\'on (entrada, salida, \emph{precondic\'on} y \emph{poscondici\'on}) y un algoritmo que las realice.
\begin{enumerate} \small
\item Comprar una lata de gaseosa en una m\'aquina.
\item Colgar un cuadro.
\item Hacer milanesas.
\item Saber cuantas letras \textit{i} tiene una palabra.
\item Encontrar el mayor valor de una lista de n\'umeros.
\item Saber si una palabra es un pal\'indromo (\emph{capic\'ua}). 
\item Buscar una palabra en un diccionario.
\end{enumerate}
\end{exercise}

\end{document}

