
\usepackage[spanish]{babel}
\usepackage[ansinew]{inputenc}
\usepackage{amsmath}
\usepackage{enumerate}
\usepackage{latexsym}
\usepackage[noend]{algorithmic}
\usepackage{algorithm}

\title{Algoritmos de Ordenamiento}
\author{Algoritmos y Estructuras de Datos \\
Programaci�n Estructurada}
\date{}

\newcommand{\pseudohrule}{\vskip 5pt \hrule}

\newenvironment{pseudo}[1][\normalsize]{
\begin{figure}
#1
\begin{semiverbatim}
}{
\end{semiverbatim}
\normalsize
\end{figure}
}

\setlength{\parskip}{3pt plus 1pt minus 1pt}

\begin{document}

\begin{frame}
\titlepage
\end{frame}

%%%%%%%%%%%%%%%%%%%%%%%%%%

\begin{frame}[fragile]{Repaso}

\begin{block}<1->{Algoritmo}
Un algoritmo es una lista de finita intrucciones que permiten hallar la soluci�n a un problema.
\end{block}

\begin{block}<2->{Especificaci�n}
La especificaci�n de un algoritmo es un contrato entre el programador y el usuario en donde se definen 
cuales son los datos de entrada, la precondici�n y la postcondici�n del mismo.
\end{block}

\begin{block}<3->{Correctitud}
Un algoritmo es correcto cuando siempre que vale la precondici�n podemos asegurar que el programa termina y la postcondici�n se cumple.
\end{block}

\end{frame}

%%%%%%%%%%%%%%%%%%%%%%%%%%

\begin{frame}[fragile]{Enunciado}

Dado un arreglo, queremos ordenarlo de manera ascendente.

\begin{pseudo}
arreglo = [5,2,7,3,7,4,9,1,8,0,2]
sort(arreglo)
print arreglo
> [0,1,2,2,3,4,5,7,7,8,9]
\end{pseudo}

\uncover<2>{Vamos a estudiar distintos \textbf{algoritmos} que nos permiten realizar esto.}

\end{frame}

%%%%%%%%%%%%%%%%%%%%%%%%%%

\begin{frame}
\begin{figure}
\huge
\centering
Insertion sort
\end{figure}
\end{frame}

%%%%%%%%%%%%%%%%%%%%%%%%%%

\begin{frame}[fragile]{Insertion sort}

Insertion sort mantiene ordenado el principio del arreglo. Al avanzar sobre un nuevo elemento, lo inserta ordenado en la parte ya ordenada, y luego prosigue con el siguiente.

\begin{pseudo}
Insertion sort
\pseudohrule
para cada posicion \textit{i} del arreglo
  insertar ordenado arreglo[i] en [0..i]
fin
\end{pseudo}

\end{frame}

%%%%%%%%%%%%%%%%%%%%%%%%%%

\begin{frame}[fragile]{Insertando}

Podemos revisar la funcion insertar ordenado para que aproveche que el elemento ya est� en el arreglo, y haga los movimientos de elementos al mismo tiempo que busque la posici�n donde va el nuevo elemento.

\begin{pseudo}
Inserta el ultimo elemento ordenado en el intervalo
Entrada: \textit{arreglo}, entero \textit{desde}, entero \textit{hasta}
Pre:  \textit{desde} y \textit{hasta} son posiciones del arreglo, desde <= hasta
      \textit{arreglo[desde..hasta-1]} se encuentra ordenado
Post: \textit{arreglo[desde..hasta]} se encuentra ordenado
\pseudohrule
int i = hasta
mientras desde < i y arreglo[i] < arreglo[i-1]
  intercambiar i, i-1
  decrementar i
fin
\end{pseudo}

\end{frame}

%%%%%%%%%%%%%%%%%%%%%%%%%%

\begin{frame}[fragile]{Intercambiando}

Por �ltimo, especificamos e indicamos el c�digo de la funci�n \texttt{intercambiar}.

\begin{pseudo}[\small]
Intercambia los elementos de dos posiciones del arreglo
Entrada: \textit{arreglo}, entero \textit{i}, entero \textit{j}
Pre:  \textit{i} y \textit{j} son posiciones del arreglo
Post: \textit{arreglo[i]} es igual a \textit{arreglo[j]} al entrar al procedimiento
      \textit{arreglo[j]} es igual a \textit{arreglo[i]} al entrar al procedimiento
\pseudohrule
temp = arreglo[i]
arreglo[i] = arreglo[j]
arreglo[j] = temp
\end{pseudo}

\end{frame}

%%%%%%%%%%%%%%%%%%%%%%%%%%

\begin{frame}[fragile]{Invariante}

Como siempre, y m�s ahora que los algoritmos van cobrando complejidad, identificamos el invariante de ciclo de insertion sort.

\begin{pseudo}
Insertion sort
\pseudohrule
para cada posicion \textit{i} del arreglo
  insertar ordenado arreglo[i] en [0..i]
fin
\end{pseudo}

\begin{block}{Invariante}<2->
En la iteraci�n $i$ del ciclo, los elementos de la posici�n $0$ hasta $i$ inclusive est�n ordenados entre s�.
\end{block}

\end{frame}

%%%%%%%%%%%%%%%%%%%%%%%%%%

\begin{frame}[fragile]{Propiedades}

\begin{itemize}
\item No requiere espacio adicional
\item Es estable, respeta el orden previo entre elementos 'iguales'
\item Realiza muchas comparaciones \uncover<2>{$(n^2)$} e intercambios \uncover<2>{$(n^2)$}
\item Es adaptativo, realiza menos operaciones si el arreglo ya est� ordenado
\end{itemize}

\end{frame}

%%%%%%%%%%%%%%%%%%%%%%%%%%

\begin{frame}
\begin{figure}
\huge
\centering
Selection sort
\end{figure}
\end{frame}

%%%%%%%%%%%%%%%%%%%%%%%%%%

\begin{frame}[fragile]{Selection sort}

Selection sort busca el elemento m�s chico del arreglo, y lo ubica en la primer posici�n. Luego, busca el m�s chico de los restantes para ubicarlo en la segunda posici�n, y as� sucesivamente hasta ordenar todo el arreglo.

\begin{pseudo}
Selection sort
\pseudohrule
para cada posicion \textit{i} del arreglo
  int k = indice del elemento mas chico en [i..len(arreglo)-1]
  intercambiar(\textit{i}, \textit{k})
fin
\end{pseudo}

\end{frame}

%%%%%%%%%%%%%%%%%%%%%%%%%%

\begin{frame}[fragile]{Auxiliares}

Como siempre, si hacemos uso de otras funciones que no sean primitivas, tenemos que definirlas.

\begin{pseudo}[\small]
Devuelve el indice del elemento m�s chico en el intervalo del arreglo
Entrada: \textit{arreglo}, entero \textit{desde}, entero \textit{hasta}
Salida:  entero \textit{i}
Pre:  \textit{desde} y \textit{hasta} son posiciones del arreglo, desde <= hasta
Post: \textit{arreglo[i]} es menor o igual a todo otro elemento del arreglo 
      entre \textit{desde} y \textit{hasta}
\pseudohrule
minimo = desde
para i = desde .. hasta
  si arreglo[i] < arreglo[minimo]
    minimo = i
  fin
fin
devolver minimo
\end{pseudo}

\end{frame}

%%%%%%%%%%%%%%%%%%%%%%%%%%

\begin{frame}[fragile]{Invariante}

Veamos el invariante de selection sort:

\begin{pseudo}
Selection sort
\pseudohrule
para cada posicion \textit{i} del arreglo
  int k = indice del elemento mas chico en [i..len(arreglo)-1]
  intercambiar(\textit{i}, \textit{k})
fin
\end{pseudo}

\begin{block}{Invariante}<2->
En la iteraci�n $i$ del ciclo, los elementos de la posici�n $0$ hasta $i$ inclusive est�n ordenados entre s�. M�s a�n, est�n en sus posiciones finales.
\end{block}

\end{frame}

%%%%%%%%%%%%%%%%%%%%%%%%%%

\begin{frame}[fragile]{Propiedades}

\begin{itemize}
\item No requiere espacio adicional
\item No es estable, no respeta el orden entre elementos 'iguales'
\item Realiza muchas comparaciones \uncover<2>{$(n^2)$} 
\item Pero pocos intercambios \uncover<2>{$(n)$}
\end{itemize}

\end{frame}

%%%%%%%%%%%%%%%%%%%%%%%%%%

\begin{frame}
\begin{figure}
\huge
\centering
Bubble sort
\end{figure}
\end{frame}

%%%%%%%%%%%%%%%%%%%%%%%%%%

\begin{frame}[fragile]{Bubble sort}

En bubble sort los elementos m�s chicos 'burbujean' hacia el comienzo del arreglo. Consiste en varias pasadas sobre el arreglo: en cada una, se intercambian dos elementos consecutivos que estuvieran desordenados. El algoritmo termina cuando el arreglo est� ordenado, lo que sucede en a lo sumo $n$ pasadas.

\begin{pseudo}
Bubble sort
\pseudohrule
para \textit{i} = 0 .. n-1
  para \textit{j} = n-1 .. i
    si arreglo[j] < arreglo[j-1]
      intercambiar j, j-1
    fin
  fin
fin
\end{pseudo}

\end{frame}

%%%%%%%%%%%%%%%%%%%%%%%%%%

\begin{frame}[fragile]{Invariante}

Observemos que el invariante es el que nos permite que el ciclo interno recorra menos posiciones.

\begin{pseudo}
Bubble sort
\pseudohrule
para \textit{i} = 0 .. n-1
  para \textit{j} = n-1 \alert{.. i}
    si arreglo[j] < arreglo[j-1]
      intercambiar j, j-1
    fin
  fin
fin
\end{pseudo}

\begin{block}{Invariante}<2->
En la iteraci�n $i$ del ciclo, los elementos de la posici�n $0$ hasta $i$ inclusive est�n ordenados entre s�, y est�n en sus posiciones finales.
\end{block}

\end{frame}

%%%%%%%%%%%%%%%%%%%%%%%%%%

\begin{frame}[fragile]{Propiedades}

\begin{itemize}
\item No requiere espacio adicional
\item Es estable, respeta el orden previo entre elementos 'iguales'
\item Realiza muchas comparaciones \uncover<2>{$(n^2)$} e intercambios \uncover<2>{$(n^2)$}
\item Es adaptativo, realiza menos operaciones si el arreglo ya est� ordenado
\end{itemize}

\end{frame}

%%%%%%%%%%%%%%%%%%%%%%%%%%

\begin{frame}[fragile]{Referencias}

\begin{itemize}
\item Animaciones, pseudoc�digo e invariantes (en ingl�s) \\
http://www.sorting-algorithms.com/
\end{itemize}

\end{frame}

%%%%%%%%%%%%%%%%%%%%%%%%%%

\end{document}

