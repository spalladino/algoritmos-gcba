\documentclass[11pt,a4paper]{article}

\pagestyle{empty}

\usepackage{a4wide}
\usepackage{enumerate}
\usepackage{amssymb}
\usepackage[ansinew]{inputenc}
\usepackage{algorithmic}
\usepackage{latexsym}
\usepackage{algorithm}                              
\usepackage[spanish]{babel}
\usepackage{amsmath}
\usepackage{listings}

\renewcommand{\algorithmicwhile}{\textbf{mientras}}
\renewcommand{\algorithmicdo}{\textbf{hacer}}
\renewcommand{\algorithmicend}{\textbf{fin}}
\renewcommand{\algorithmicreturn}{\textbf{devolver}}
                                                     
\title{Examen Final}
\author{Curso de Algoritmos y Estructuras de Datos}
\date{17 de Febrero de 2012}

\setlength{\textwidth}{17cm}
\setlength{\textheight}{24cm}
\setlength{\topmargin}{-2cm}               % distance from top of page to begining of text
\setlength{\topskip}{0cm}                  % between header and text
\setlength{\oddsidemargin}{-0.5cm}         % odd page left margin
\setlength{\evensidemargin}{-0.5cm}        % even page left margin

\begin{document}
\maketitle
%\textbf{Justifique todas sus respuestas.} \vskip 5pt
\hrule
\vskip 5pt

\begin{enumerate}
\item Dado el siguiente algoritmo, que toma como entrada dos enteros positivos $x$ e $y$:
\begin{algorithmic}
\WHILE {$x > 0$}
  \STATE decrementar $x$ en $1$
  \STATE incrementar $y$ en $1$
\ENDWHILE
\RETURN $y$
\end{algorithmic}

Se pide:
\begin{enumerate}
\item Indicar qu� valor devuelve el algoritmo cuando se lo invoca con $x = 2, y = 4$.
\item Explicar en palabras qu� hace el algoritmo (notar la relaci�n entre los valores de entrada y el de salida, hacer m�s ejemplos de ser necesario).
\item Dar un nombre adecuado al algoritmo en funci�n del item anterior.
\item Qu� sucede cuando se invoca al algoritmo con $x = -3, y = 2$?
\item Completar entrada, salida, precondici�n y postcondici�n del algoritmo.
\item Dar el invariante para el �nico ciclo que tiene el algoritmo.
\end{enumerate}

\item Responda las siguientes preguntas:
\begin{enumerate}
\item Qu� es un algoritmo?
\item Qu� es la especificaci�n de un algoritmo?
\item Cu�ndo un algoritmo es correcto?
\item Cu�l es la diferencia entre una funci�n y un procedimiento?
\end{enumerate}

\item \textbf{Justifique} qu� estructura de datos usar�a en cada uno de los siguientes casos:
\begin{enumerate}
\item Guardar la temperatura de 10 ciudades durante una semana.
\item Mantener un listado de todos los candidatos que se van postulando en una elecci�n.
\end{enumerate}

\item Indique cu�l es el siguiente algoritmo de ordenamiento:
\begin{lstlisting}[language=Python]
def sort(array):
    for i in range(0, len (array)):
        min = i
        for j in range(i + 1, len(array)):
            if array[j] < array[min]:
                min = j
        array[i], array[min] = array[min], array[i]
\end{lstlisting}

\end{enumerate}

\end{document}